\documentclass[11pt]{article}

\usepackage[utf8]{inputenc}
\usepackage[scale=0.75,a4paper]{geometry}
\usepackage[german]{babel}
%opening
\title{InformatiCup - Dokumentation}
\author{Tilman Hinnerichs, Tobias John}

\begin{document}

\maketitle

\begin{abstract}
	Im Folgenden soll unsere Lösung der Aufgaben des 13. InformatiCup 2018 vorgestellt werden. Dabei möchten wir unsere Annahmen zur Aufgabenstellung, unsere Lösung für beide Probleme und deren Bewertung, und einen Ausblick unserer Lösung beschreiben und erklären.  
\end{abstract}

\section{Annahmen zur Aufgabenstellung und Umformungen der Eingabedaten}
	Für Umsetzung unserer Lösung mussten wir dafür sorgen, dass die Werte nicht nur eingelesen, sondern auch so umgeformt werden, dass diese bestimmte Kriterien erfüllen.  
\subsection{Umformung der Tankstellendaten}
	Die Datei der Tankstellendaten stellt sehr viele unterschiedliche Daten zur Verfügung. Die reichen von der ID der Tankstelle, über die Postleitzahl zu den Koordinaten der Tankstelle. Für unser Verfahren ist sind folgende Kriterien wichtig: \\
	\begin{enumerate}
		\item \textbf{Individualität} der Tankstellen um diese eindeutig zuordnen und um über diese in einer Liste iterieren zu können
		\item \textbf{Örtliche Zuordnung} der Tankstellen um zwischen ihnen Abstände zu errechnen
	\end{enumerate} 
	Um eventuell später Zusammenhänge einzubeziehen oder wiederzuerkennen, haben wir zusätzlich die Marke der einzelnen Tankstellen mit einbezogen. Dies bildet aber zur Erfüllung der obigen Kriterien keinen Mehrwert. \\
	So werden bei der Einbeziehung der Tankstellendaten nur deren ID, Marke und Koordinaten aus den Tankstellendaten aus oben genannten Gründen herausgelesen. 
\subsection{Umformung der historischen Benzinpreisdaten}
	Die Daten, die wohl am wichtigsten für die Voraussage der Benzinpreise selbst sind, sind die historischen Tankstellendaten. Bei diesen stehen Tausende von Daten pro Tankstelle zur Verfügung. Pro Änderung des Benzinpreises stehen dabei das Datum mit der Uhrzeit mit einer Genauigkeit bis auf die Sekunde genau und der zugehörige neue E5-Benzinpreis zur Verfügung. 
	\\
	Kriterien für die Umformung dieser Daten sind die folgenden:
	\begin{enumerate}
		\item \textbf{Vergleichbarkeit} der Datensätze um mit ihnen eine Einteilung in bestimmte Kategorien vorzunehmen
		\item \textbf{Kontinuität} der Daten um eine Vergleichbarkeit der Daten bestimmter gleicher Intervalle zu erreichen und um diese in Algorithmen einfüttern zu können
		\item \textbf{Diskretisierung} der gemessenen Zeitpunkte
		\item \textbf{Handhabbare Menge} an Daten um mit ihnen noch in annehmbarer Zeit rechnen zu können, aber ebenfalls noch so viel, dass nicht zu viel Information verloren geht
	\end{enumerate}

	Um eine Vergleichbarkeit der Datensätze zu gewährleisten mussten wir die Kontinuität der Daten erreichen. Dazu rundeten wir jeden der Änderungszeitpunkte auf seine Stunde herunter und rechneten dabei ebenfalls die zugehörige Zeitverschiebung mit ein. Ebenfalls reicherten wir die Daten mit den Preisen zwischen diesen Änderungszeitpunkten an. Was auf den ersten Blick recht trivial wirkt, nimmt bei weiteren Algorithmen sehr viel Arbeit ab. So beträgt beispielsweise der Benzinpreise nach einer Änderung auf ein bestimmtes Niveau bis zu seiner nächsten Änderung diesen Wert bei. Durch diese Anreicherung besteht ein jeder Tag von Werten in der Geschichte einer Tankstelle aus 24 Werten. Dies bietet ebenfalls eine wunderbare Unabhängigkeit von schnell schwankenden Preisen innerhalb eines Tages und eben die oben gewünschte Vergleichbarkeit, da nun die Werte eines Tages mit 24 Werten dargestellt werden können. Damit ist zugleich die Kontinuität durch die gleichen Intervalle zwischen den Messpunkten und die Diskretisierung durch 24 feste Punkte innerhalb der sonst linearen Zeit gegeben. \\ Abschließend ist aus unserer Sicht das dilemmaartige vierte Kriterium der handhabbaren Menge an Daten gewährleistet. So sind durch die Einteilung noch eine Menge von $ 24 * 365 * 3 = 26.280 $ Werten (bei ca. 3 betrachteten Jahren der historischen Tankstellenwerte) vorhanden, was eine ausreichende Genauigkeit für zukünftige Voraussagen mit unserer Methode bietet und eine für Computer komfortable Anzahl an Werte bildet.
		
\subsection{Umformung der Routendaten}
	Die Routendaten sind vor allem für die Berechnung der besten Route von Bedeutung. Diese enthalten neben der maximalen Tankkapazität des Fahrzeugs auch die Ankunftszeiten an den bestimmten Tankstellen, sowie deren ID. Um hier die Kausalität der bisherigen Daten weiterzuführen, diskretisieren wir wieder die Daten auf die volle Stunde und verwerfen für unser Modell zu spezifische  Daten wie Minuten und Sekunden. Weitere Annahmen müssen dabei für die Routendaten nicht getätigt werden, um sie in Konformität zu bringen.
\subsection{Umformung der Preisvorhersagedaten}
	Die getätigten Annahmen zu den Preisvorhersagedaten sind ähnlichen zu den unter \glqq Umformung der Routendaten\grqq{} getätigten Annahmen. So wird wieder sowohl der Zeitpunkt, welcher als zuletzt bekannt angenommen werden soll, als auch der Zeitpunkt, bis zu welchem die Vorhersage reichen soll, auf die jeweilige Stunde gerundet, um in das oben angerissene Datenmodell zu passen.

\section{Algorithmenidee}
\subsection{Lösung der Benzinpreis-Vorhersage-Problems}
\subsubsection{Ansatz}
	Welche Algorithmen wurde hier benutzt uns was versprechen wir uns davon? Warum haben wir das ganze in Intervalle unterteilt? Gehen wir auf etwaige Sonderdaten wie in der Aufgabenstellung genannt ein (Schulferien, Adresse,) 
	\newline
	\newline
	Wir sollen das ganze an folgenden vergleichbaren Schritten herleiten und erklären:
	\begin{enumerate}
		\item \begin{quote}
			Entscheiden Sie, ob Sie ein Vohersagemodell für alle Tankstellen oder individuelle Vorhersagemodelle für einzelne Tankstellen oder für Klassen von öhnlichen Tankstellen entwickeln wollen. Überlegen Sie sich dazu den möglichen Einfluss von Mermalen einer Tankstelle auf die Entwicklung ihrer Benzinpreise (z.B. Tankstellen an Autobahnen, Tankstellen die vom Berufsverkehr erfasst werden oder Tankstellen in wenig erschlossenen Regionen). Dokumentieren Sie ihre Ergebnisse.
		\end{quote}
		\item Zusätzliche Informationen?
		\item Diskutieren der Prognoseergebniss
	\end{enumerate}
	
	Und zusätzlich:
	\begin{quote}
		Überlegen Sie im nächsten Schritt, welche zusätzlichen Informationen wie zum Beispiel die Wochentage, Ferienzeiten oder Verkehrsinformationen für Ihre Benzinpreisvorhersagen sinnvoll sein könnten. Erweitern Sie Ihre Vorhersagemodelle entsprechend und dokumentieren Sie Ihre Ergebnisse.
	\end{quote}
\subsubsection{Umsetzung der Lösung}
	Wie haben wir diese Idee umgesetzt? Warum haben wir wie viele Dimensionen an die SOM/SOFM vergeben? Warum halten wir das für sinnvoll?
\subsubsection{Bewertung der Lösung}
	Hier könnte Ihr Diagramm stehen. \\
	Wie gut ist das was wir da gebastelt haben?
	\begin{quote}
		Diskutieren Sie die Güte Ihrer erzielten Prognoseergebnisse für den geforederten Vorhersagezeitraum (d.h. bis zu einem Monat in die Zukunft). Verwenden Sie dazu geeignete Maßzahlen für die Güte Ihrer Vorhersagemodelle. 
	\end{quote}
\subsection{Lösung des Routenproblems}
\subsubsection{Ansatz der Lösung}
	Für die Lösung des Routenproblems wurde der in der Aufgabenstellung beschriebene Algorithmus \glqq To fill or not to fill\glqq{} benutzt. Der Ansatz allein bedarf deswegen keiner weiteren Erläuterung, wobei unsere Umsetzung hingegen erklärenswert ist.
\subsubsection{Umsetzung der Lösung}
	Das Problem wurde dafür in mehrere Unterprobleme unterteilt, die sich wunderbar in einer Softwarelösung umsetzen lassen. So wurde dieses Problem in die Teilprobleme der Entfernungsberechnung, Preisfindung für eine spezifische Tankstelle auf der Route und den Algorithmus selbst untergliedert. \\
	Für die Entfernungsberechnung wurde dabei die Formel für die Entfernung auf Großkreisen verwendet. Um von einer Tankstelle zu einer anderen Tankstelle zu fahren, wobei sich strikt an die vorgegebene Route gehalten werden muss wurde diese Entfernungsberechnung rekursiv aufgebaut. Eine Strecke von Knoten A zu Knoten B erfolgt demnach über die Berechnung und Summierung der Einzelstrecken zu den Knoten zwischen den A und B. Andere Methoden wie das direkte Fahren von A zu B, wenn dies der aktuelle Benzinstand zulässt, würde bei Routen wie der vorgegebenen Bertha-Benz-Memorial-Route, welche eine Rundreise darstellt, zu unerwünschten Effekten führen. \\
	Für das Auffinden der bereits errechnete Preise müssen diese bereits berechnet vorliegen. \\
	Der Algorithmus selbst bildet dabei das Kernstück der Tankstrategie und wurde wie in dem Paper gegeben umgesetzt. Dafür müssen die folgenden Teilprobleme wie im Kapitel der Implementierung beschrieben umgesetzt werden:
	\begin{enumerate}
		\item \textbf{Die Next-Funktion}, welche wie beschrieben dafür sorgt den billigsten nächsten Knoten zu finden.
		\item \textbf{Die Previous-Funktion}, welche den billigsten zurückliegenden Knoten liefert. Falls keiner billiger sein sollte, wird der Startknoten selbst ausgegeben.
		\item \textbf{Finden aller Breakpoints}, also aller Knoten die keinen Knoten hinter sich in Reichweite haben, sodass dieser einen billigeren Preis liefert.
		\item \textbf{Fahren zum nächsten Knoten}, also das schließliche Weiterfahren, mit Berechnung der Verbrauchs und Berechnung der nachzutankenden Menge.
	\end{enumerate}
	Wobei die beschreibenden Formeln in der Beschreibung des Algorithmus zu finden sind.
\section{Implementierung und Umsetzung der Lösung}
	Hier könnte Ihr Klassendiagramm stehen.	
	Entwurf und Struktur der Lösung
\subsection{Klasse GasStation}
	Welche Aufgaben hat die Klasse zu erfüllen? Was sind die wichtigsten Funktionen und was tun diese? Wie werden die Datenstrukturen befüllt und warum auf diese Weise?

	Welche der vielen, vorhandenen Daten benutzen wir überhaupt? Warum können wir den Rest verwerfen?
\subsection{Klasse Route}
	Welche Aufgaben hat die Klasse zu erfüllen? Was sind die wichtigsten Funktionen und was tun diese? Wie werden die Datenstrukturen befüllt und warum auf diese Weise?
\subsection{Klasse PrizeForecast}
	Welche Aufgaben hat die Klasse zu erfüllen? Was sind die wichtigsten Funktionen und was tun diese? Wie werden die Datenstrukturen befüllt und warum auf diese Weise?
\subsection{Klasse Supervisor}
	Welche Aufgaben hat die Klasse zu erfüllen? Was sind die wichtigsten Funktionen und was tun diese? Wie werden die Datenstrukturen befüllt und warum auf diese Weise?
\subsection{Klasse Model}
	Welche Aufgaben hat die Klasse zu erfüllen? Was sind die wichtigsten Funktionen und was tun diese? Wie werden die Datenstrukturen befüllt und warum auf diese Weise?
\subsection{Klasse Strategy}
	Welche Aufgaben hat die Klasse zu erfüllen? Was sind die wichtigsten Funktionen und was tun diese? Wie werden die Datenstrukturen befüllt und warum auf diese Weise?

\section{Ausblick und Erweiterungen}
	InformatiCup sagt, dass man tolle Erweiterungen einfügen könnte. Ein kleiner Auszug: 
	\begin{quote}
		Falls nach der erfolgreichen Implementierung der Grundanforderungen noch Zeit für Erweiterungen bleibt, seien Sie kreativ, zum Beispiel mit einer mobilen App für echte Benzinpreisvorhersagen unterwegs. Oder passen Sie Ihre Softwareanwendung für die Ausführung in der Cloud für eine hohe Performanz, Skalierbarkeit und Verfügbarkeit an. Sie möchten nicht mit einem durchschnittlichen Kraftstoffverbrauch rechnen? Integrieren sie mögliche Spezialsoftware für die Simulation des Benzinverbrauchs. Oder verwenden Sie Daten aus Online-Kalendern als Grundlage für Ihre Fahrzeugrouten. Seien Sie kreativ!
	\end{quote}
	Uns weiterhin:
	\begin{quote}
		... Ausblick: Lässt sich Ihr Verfahren vielleicht auf auf Ladevorgänge in der Elektromobilität oder für ganz andere Aufgabenstellungen anwenden?
	\end{quote}
	
\end{document}
