\documentclass[11pt]{article}

\usepackage[utf8]{inputenc}
%opening
\title{InformatiCup - Dokumentation}
\author{Tilman Hinnerichs, Tobias John}

\begin{document}

\maketitle

\begin{abstract}
	Im Folgenden soll unsere Lösung der Aufgaben des 13. InformatiCup 2018 vorgestellt werden. Dabei möchten wir unsere Annahmen zur Aufgabenstellung, 
\end{abstract}

\section{Annahmen zur Aufgabenstellung}
\section{Lösungsansatz}
\subsection{Entwurf und Struktur der Lösung}
\subsubsection{Klasse GasStation}
	Welche Aufgaben hat die Klasse zu erfüllen? Was sind die wichtigsten Funktionen und was tun diese? Wie werden die Datenstrukturen befüllt und warum auf diese Weise?
	
	Welche der vielen, vorhandenen Daten benutzen wir überhaupt? Warum können wir den Rest verwerfen?
\subsubsection{Klasse Route}
	Welche Aufgaben hat die Klasse zu erfüllen? Was sind die wichtigsten Funktionen und was tun diese? Wie werden die Datenstrukturen befüllt und warum auf diese Weise?
\subsubsection{Klasse PrizeForecast}
	Welche Aufgaben hat die Klasse zu erfüllen? Was sind die wichtigsten Funktionen und was tun diese? Wie werden die Datenstrukturen befüllt und warum auf diese Weise?
\subsubsection{Klasse Supervisor}
	Welche Aufgaben hat die Klasse zu erfüllen? Was sind die wichtigsten Funktionen und was tun diese? Wie werden die Datenstrukturen befüllt und warum auf diese Weise?
\subsubsection{Klasse Model}
	Welche Aufgaben hat die Klasse zu erfüllen? Was sind die wichtigsten Funktionen und was tun diese? Wie werden die Datenstrukturen befüllt und warum auf diese Weise?
\subsubsection{Klasse Strategy}
	Welche Aufgaben hat die Klasse zu erfüllen? Was sind die wichtigsten Funktionen und was tun diese? Wie werden die Datenstrukturen befüllt und warum auf diese Weise?
\subsection{Ansatz zur Lösung der Benzinpreis-Vorhersage-Problems}
	Welche Algorithmen wurde hier benutzt uns was versprechen wir uns davon? Warum haben wir das ganze in Intervalle unterteilt? Gehen wir auf etwaige Sonderdaten wie in der Aufgabenstellung genannt ein \(Schulferien, Adresse,\)
\section{Ausblick und Erweiterungen}
	InformatiCup sagt, dass man tolle Erweiterungen einfügen könnte. Ein kleiner Auszug: 
	\begin{quote}
		Falls nach der erfolgreichen Implementierung der Grundanforderungen noch Zeit für Erweiterungen bleibt, seien Sie kreativ, zum Beispiel mit einer mobilen App für echte Benzinpreisvorhersagen unterwegs. Oder passen Sie Ihre Softwareanwendung für die Ausführung in der Cloud für eine hohe Performanz, Skalierbarkeit und Verfügbarkeit an. Sie möchten nicht mit einem durchschnittlichen Kraftstoffverbrauch rechnen? Integrieren sie mögliche Spezialsoftware für die Simulation des Benzinverbrauchs. Oder verwenden Sie Daten aus Online-Kalendern als Grundlage für Ihre Fahrzeugrouten. Seien Sie kreativ!
	\end{quote}
	
\end{document}
